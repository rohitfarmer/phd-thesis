\chapter{Appendix IV}
\label{cha:AppendixIV}
%\addcontentsline{toc}{chapter}{\quad\,\,{Appendix II}}
\begin{doublespacing}

		\section{Computational Tools available for the PKS researcher.}
		\label{sec:comptools}%		
		Comparing all the resources mentioned in Section \ref{sec:BioinfoPKS} the various tasks that can be performed can be divided into four major categories: 1) obtaining a well curated PKS/NRPS cluster database for further analysis; 2) domain detection for various modular PKSs and NRPs; 3) predicting substrate specificity for various starter and extender units; and 4) correlating identified clusters to their corresponding metabolites.
		
		The NRPS-PKS database serves as an amalgamated source for PKS clusters (both modular and iterative), NRPS clusters and chalcone like products. At the time of first publication data in this database was derived from the PKSDB (19 modular PKS cluster), NRPSDB (17 NRPS clusters and 5 hybrid PKS+NRPS clusters), ITERDB (21 iterative PKSs) and CHSDB (11 plant chalcone PKSs and 3 bacterial chalcone PKSs) databases \parencite{Yadav2003a}. Another database ASMPKS at the time of publication contained 41 characterized PKS pathways including everything in PKSDB, with more entries being added \parencite{Tae2007}. Users may also add or delete their own entries. NORINE is a database for non-ribosomal peptides containing 1122 peptide products and over 500 monomers as of 04/2010, however it does not provide information on biosynthesis \parencite{Caboche2008}.
		
		Why is there a need for specialized databases for PKS/NRPS research when databases like CDD \parencite{Marchler-Bauer2011} and interPro \parencite{Hunter2012}, or domain finding software like SMART \parencite{Letunic2012} exist? It was observed by \textcite{Yadav2003} during the construction of PKSDB/SEARCHPKS that inspite of CDD and interPro being a vast source of protein domains they suffer from being general and not tailored for a specific purpose. Comparing the CDD results from their domain identification program they found that at that time CDD failed to detect any of the DH domains in the modular PKS clusters and was also not able to distinguish between the KR and ER domains. However, over the time CDD has improved and in my analysis with the MmpD subunit from the mupirocin pathway it is able to predict all the domains except the catalytically inactive DH domain in module 1.
		
		Tailored NRPS/PKS domain prediction has been achieved by either a BLAST \parencite{Altschul1990} search of the query sequence against a backend database or through querying a database of profile hidden markov models (HMMs) trained on domains from PKS/NRPS clusters. The domain detection algorithms in NRPS-PKS, ASMPKS and SBSPKS use BLAST whereas ClustScan, NRPSsp, CLUSEAN and antiSMASH uses profile HMMs. HMMs have also been used by Ansari and coworkers \parencite{Ansari2008} to identify methyl transferase (MT) domains present in type I PKS and NRPS megasynthases and to sub-group them into N-MT, C-MT and O-MT groups based on the prediction of their site of methylation. In another study,  \textcite{Foerstner2008} used HMMs to screen eight metagenomics shot gun data sets in order to estimate the frequency of type I PKSs, using HMMs for eight domains. They also incorporated analysis of maximum-likelihood phylogenetic trees to increase the reliability and resolution of the dataset and to discriminate true PKS I domains from evolutionarily related but functionally different ones. 
		
		ClustScan \parencite{Starcevic2008} also utilizes profile HMMs, both extracted from Pfam and specifically constructed profiles. ClustScan works as a client server application with the main program running on a linux server and a Java client running on the user's computer, making it compatible with Windows, Mac OS and Linux operating systems. It uses Glimmer or GeneMark for the gene prediction followed by HMMs for the domain identification in PKS/NRPS/PKS-NRPS hybrid clusters. At the time of publication ClustScan database contained data for 57 PKS clusters, 51 NRPS clusters and 62 PKS-NRPS hybrid clusters \parencite{Starcevic2008}. Apart from domain identification and substrate specificity detection ClustScan can also exports the chemical structures of predicted products in a SMILES/SMARTS format for further analysis by standard chemistry programs. ClustScan also acts as a graphical interface to CompGen, a tool that undertakes \textit{in silico} homologous recombination of PKS gene clusters, predicting whether a particular recombination is likely to be functional, and the polyketide product to be expected \parencite{Starcevic2012}. ClustScan can be accessed using a 30-day evaluation license; the database behind ClustScan, ClustScanDB is freely available via a web interface (\url{http://csdb.bioserv.pbf.hr/csdb/ClustScanWeb.html}).
		
		To facilitate the systematic mapping of secondary metabolites in fungal genomes \textcite{Khaldi2010} developed the SMURF (Secondary Metabolite Unknown Region Finder). SMURF is a web based tool which relies on hidden Markov model searches against Pfam and TIGRFAMs \parencite{Haft2001} domains to detect backbone genes in sequenced fungal genomes. SMURF is not only capable of predicting backbone genes but also tailoring enzymes and \textcite{Khaldi2010} used SMURF to catalogue putative clusters in 27 publically available fungal genomes. They also compared the predicted results with genetically characterized clusters from 6 fungal species and demonstrated that SMURF is capable of predicting accurately all of these clusters. 
		
		Recently NRPSpredictor2 \parencite{Rottig2011} used an innovative method employing a machine learning algorithm called a transductive support vector machine (SVM) for predicting the specificity of adenylation domains in NRPs from amino acid sequence data. The method is based on previous work \parencite{Rausch2005} from the same group however the new version outperforms the previous version by predicting the specificity of adenylation domains at four hierarchical levels, ranging from gross physicochemical properties of an A-domain's substrate to single amino acid substrates as well as predicting A-domain specificity in fungal systems, which was not achieved in the previous version. The NRPSpredictor2 utilizes the active site lining residues within 8 \AA{} of the bound phenylalanine ligand in the crystal structure (PDB ID 1AMU) of the peptide synthetase gramicidin S synthetase 1 (GrsA). These 34 extracted positions were then located in the A-domain sequences of the training data set, and this data was input into the SVM to train predictors of substrate specificity. The NRPSpredictor2 also has a larger database of training data including the sequences from fungal counterparts, as compared to its previous version, thus enabling a wider and more accurate prediction rate. This work is an extension of ideas built up by a number of researchers over a number of years, as discussed further below.
		
		Many research groups have used structural data to determine conserved residues lining the active site, which they then use for the prediction of substrate specificity. In the case of PKS this is typically specificity for extender or starter units, in the case of NRPs specificity for amino acids. Prior to NRPSpredictor2, Stachelhaus and coworkers \parencite{Stachelhaus1999} had utilized the 10 active site lining residues from the same crystal structure of the peptide synthetase gramicidin S synthetase 1 (GrsA, PDB ID 1AMU) and succeeded in predicting the specificity of the A-domain for 20 substrates. Shortly after their work Challis et al. \parencite{Challis2000}, adopting a similar strategy, extended the number of predictable NRPS substrates to 33. \textcite{Challis2000} used 8 amino acids within the binding pocket combined the phylogenetic clustering. They also performed modelling for the structures of a variety of binding pockets.
		
		Similarly, during the compilation of the first polyketide synthase database (PKSDB) and associated domain prediction program SEARCHPKS, \textcite{Yadav2003} identified 13 active site positions in AT domains required to discriminate between malonate and methylmalonate as starter and extender units in type I modular polyketide synthases. The 13 active site residues were identified on the basis of the crystal structure of acyltransferase from \textit{Escherichia coli} FAS (PDB ID 1MLA). By modelling malonate and methylmalonate in the active site of the AT they identified the residue which is responsible for controlling substrate specificity. The online server at SEARCHPKS assigns a substrate for the AT domain if all the 13 positions in the query amino acid sequence match identically to the corresponding positions in any of the AT domains found in the PKSDB database.
		
		Further utilizing structural modelling of the active site \textcite{Yadav2009} identified that certain residues in iterative KS domains can potentially control the size of final product by governing the total number of iterations. In the same work Yadav et al. also utilized profile HMMs to distinguish KS domains between modular PKSs and iterative PKSs, they also observed that HMMs are not only capable of broadly classifying KSs as modular or iterative but also of grouping them into subtypes. They proposed that such a method can help the sequencing projects as just by analyzing the KS domains of a novel PKS cluster one can identify its type and subtype and decide whether sequencing the entire cluster would be of interest. From these recent works HMMs prove to be a promising tool for various types of analysis ranging from domain identification to substrate specificity. 
		
		Recently available coordinates of crystal structures of various type I PKS catalytic domains \parencite{Keatinge-Clay2006, Khosla2007, Tang2007, Keatinge-Clay2008, Khosla2009, Tsai2009} and an almost complete module of the homo dimeric mammalian FAS protein \parencite{Maier2008} allow the modelling of PKS domains in a homo dimeric modular context, assuming the PKSs have a similar structure to the mammalian FAS \parencite{Gokhale2007, Tsai2009}. Based on this, Mohanty's group developed SBSPKS \parencite{Anand2010}. SBSPKS is a web based tool and probably the first which can model the 3D structures of a PKS module in a biologically active dimeric conformation. SBSPKS consists of three main components, MODEL\_3D\_PKS, DOCK\_DOM\_ANAL and an updated version of NRPS-PKS which was previously developed by the same group. MODEL\_3D\_PKS component models the 3D structures of a complete module of a type I modular PKS protein in dimeric form. The homology modelling protocol used involves aligning the query module sequence to the sequence of the templates by standalone BLAST and side chain modelling using SCRWL \parencite{Canutescu2003}. The templates used for the various domains of PKSs were PDB ID 2HG4, 3LE6, 1IZ0 and 2FR0 along with nine threading model based on 2FR0 for modelling structural sub-domains of KR in the cases where DH-KR and DH-ER linkers lacked homology to 2FR0. The modelled domains are then superimposed on to the corresponding mammalian FAS module to provide the relative orientation of the PKS domains in a dimeric state. MODEL\_3D\_PKS can model 3D structures for any of the four typical combinations of modular PKS i.e. KS-AT-ACP, KS-AT-KR-ACP, KS-AT-DH-KR-ACP, KS-AT-DH-ER-KR-ACP, excluding the ACP domain as there is no experimental information available to provide the relative orientation of ACP domains to the rest of the domains in the module. Although MODEL\_3D\_PKS can model the typical combinations of domains found in modular PKSs it is not designed to model the module from trans AT systems, although it may be possible to “trick” the system into modelling this. The MODEL\_3D\_PKS web interface also has an embedded JMOL applet for quick visualization of the modelled structure.
		
		The DOCK\_DOM\_ANAL module of the SBSPKS analyses the docking domains between the related subunits in a modular PKS. The docking domains are the inter-subunit linker region characterized by a four helical bundle, one helix is from the C-terminus of the preceding protein and three helices are from the N-terminus of the succeeding protein. In previous studies \parencite{Broadhurst2003, Weissman2006, Weissman2008} a \textquotedblleft docking code\textquotedblright \ was proposed, in which the electrostatic interaction between two residues in the docking domains were responsible for the inter-subunit contacts. DOCK\_DOM\_ANAL estimates the crucial inter-subunit contacts and predict the preferred order of substrate channelling utilizing the method developed by \textcite{Yadav2009}. As a point of terminology these inter subunit docking domains should not be confused with intra module segments seen in trans AT PKS I systems, which show sequence similarity to the N and C termini of \textit{cis}-AT domain, and may thus be remnants of such domains although their role is still poorly understood \parencite{Gurney2011}.
		
		The most important updated feature in NRPS-PKS component of SBSPKS is a wider range of substrate specificity detection for the AT domain. The initial version of the AT domain specificity protocol was only able to discriminate between the malonate and methyl malonate selectivity while the new version can now detect specificity for a total of 13 substrates. Another enhanced feature is its integration into the SBSPKS interface, thus providing links for automated input of its results to various other programs in the suite. 
		
		Apart from the easy-to-use web servers and client server based applications like ClustScan, recent initiatives by Weber and co-workers resulted in CLUSEAN which is \textquotedblleft a computer-based framework for the automated analysis of bacterial secondary metabolite biosynthetic gene clusters\textquotedblright \ \parencite{Weber2009}. CLUSEAN (CLUster SEquence ANalyzer) is an open source resource for semi-automatic analysis of secondary metabolite gene clusters. It is a modular framework of Bioperl \parencite{Rausch2007} programs that are compatible with LINUX, UNIX, or MS Windows systems. It currently includes BLAST and HMMER as the tools for sequence annotation and domain identification respectively. CLUSEAN scripts search the NCBI non redundant protein database with BLASTp, and use HMMER to search the following Hidden Markov Models: Pfam domains, PKS/NRPS domains and motifs, and the C-domain types and NRPS adenylation domain models developed by Rausch and co-worker \parencite{Rausch2005, Rausch2007}. %Although CLUSEAN offers a comprehensive analyses it seems difficult to operate by researchers who are not comfortable with scripts and requires manual analysis of the output.
		
		Another software pipeline probably the first and most recent of its type called antiSMASH (antibiotics and Secondary Metabolite Analysis Shell) \parencite{Medema2011} has been developed for identifying secondary metabolite biosynthesis gene clusters with the advanced features providing the analysis and annotation of the identified clusters. It serves as a meta server which amalgamates the data and methods available from various sources \parencite{Rausch2007, Ansari2008, Yadav2009, Weber2009, Letunic2009, DeJong2010, Finn2010}. It is capable of analyzing not only PKS/NRPS clusters as most of the software mentioned above do, but also for the identification of the biosynthetic loci for various other secondary metabolites as listed in Table \ref{tab:pks_res}. AntiSMASH can be accessed via a web server or it can be run as a standalone Java graphical user interface. It utilizes Glimmer3/GlimmerHMM for the gene prediction in the input sequence data and HMMER3 for the prediction of biosynthetic gene clusters using both existing profile HMMs as well as new profile HMMs from seed alignments. The substrate specificity of AT and adenylation domains were performed as proposed by \parencite{Yadav2003} and \parencite{Rottig2011} respectively along with the method proposed by \parencite{Minowa2007} for both. % Finally the predictions from all the methods are integrated into a consensus by a majority vote. 
		The stereochemistry predictions for PKSs based on the Ketoreductase (KR) domain were carried out using the method used in the program ClustScan \parencite{Starcevic2008}. To predict the biosynthetic order of PKS/NRPS modules antiSMASH uses the same method as SBSPKS to match the docking domain residues in the ORFs of type I modular PKSs, and otherwise assumes colinearity in the biosynthetic gene cluster. It also generates the SMILES string for the final predicted core chemical structure along with its picture. antiSMASH can also annotate the accessory genes by utilizing the HMMs constructed on smCOG (secondary metabolite clusters of orthologous groups). It also provides features like ClusterBlast which can be used for comparative gene cluster analysis between the queried cluster and the clusters in the database. Utilizing CLUSEAN framework modules antiSMASH can also perform various other genome-wide analysis. 



\end{doublespacing}
