\chapter*{Abstract}

Polyketide synthases (PKS) are the enzyme complexes that synthesise a wide range of natural products of medicinal interest, notably a large number of antibiotics. The present work investigated the mupirocin biosynthesis system, in comparison with similar pathways such as thiomarinol and kalimantacin, with the focus on the structural modelling of the protein complexes involved in antibiotic synthesis via the tools of structural bioinformatics. \bet - methylation in the mupirocin pathway is catalysed via enzymes encoded by the \lq\lq HMG-CoA synthase (HCS) cassette\rq\rq and replaces a \bet-carbonyl with a methyl group. To understand better, what might allow the HCS cassette to recognise \bet-branch associated ACPs, molecular modelling was used to explore the interaction of the ACPs with MupH (the HMG-CoA synthase homologue). Hidden Markov models (HMM) were used to classify ACPs as branching and non-branching. HMM analysis highlighted essential features for an ACP to behave like a branching ACP. A homology model of MupH was docked with the NMR structure of each of ACPs mupA3a and mupA3b. The docking results were also supported by the biological context based on analysis of phylogenetic variations in amino acid conservation and physical properties of the interface residues. Modelling and mutagenesis identified helix III of the ACP as a probable anchor point of the ACP:HCS complex. The position of this helix is determined by the core of the ACP and substituting the interface residues modulates the interaction specificity. 
%
%The results identified a two pin mechanism critical for the recognition specificity of the ACPs involved in the programmed \bet - branching. Colleagues from Prof. Thomas group subjected the identified residues to mutation studies, the results supports the predicted complex and the specificity of the residues involved in the \bet - branching mechanism.

Although docking and mutagenesis studies performed on ACP:MupH complex laid down the general specificity rule of ACP:MupH recognition in the \bet-branching. BatC from closely related kalimantacin cluster cannot substitute for the function of MupH. However, Prof. Thomas' group have shown upon mutating BatC it is possible to successfully complement a $ \Delta $\textit{mupH} mutant. \bet-branching ACPs from the thiomarinol cluster complements the \bet-branching ACPs in the mupirocin cluster however, the \bet-branching ACP from the kalimantacin cluster does not. These observations suggests a pair wise specificity between the ACP:HCS proteins in the \bet-branching. 

Molecular dynamics simulation of the ACP:MupH complex revealed large movements of the surface loops at the opening of the active site in MupH. These movements were found to be greater in a MupH monomer as compared to the MupH dimer structure and may assist in the accommodation of the ligand inside the MupH active site. 

Molecular dynamics simulations of apo, holo and acyl forms of ACPs in the mupirocin cluster revealed that the PKS ACPs form a cavity upon the attachment of the phosphopantetheine and acyl chains similar to what is seen in the fatty acid synthase (FAS) ACPs. The cavity formed does not form a deep tunnel as in the FAS ACPs but, is rather solvent exposed cleft, enabling the polar groups on the acyl chain to hydrogen bond with the solvent. It was also observed that a bulky residue (I61 in ACP-mupA3a) in the PKS ACPs is likely to  prohibit the formation of a deep tunnel as opposed to the case of smaller residue (alanine) at the equivalent position in the FAS ACPs, where a deep tunnel is seen.

Molecular docking of the cognate substrate with the ketosynthase (KS) homo dimer of module 5 of the MmpA in the mupirocin pathway revealed a loop at the dimer interface that appears to be responsible for the recognition specificity of $ \alpha $-hydroxylated substrate. Mutagenesis experiments showed that, upon swapping this recognition loop from a KS which does not bind an $ \alpha $-hydroxylated substrate, the pathway produces a full length product in the absence of MupA, the enzyme thought to be responsible for $ \alpha $-hydroxylation, whereas the wild type KS $ \Delta $\textit{mupA} mutant does not. Furthermore the loop swap experiment produces a product that is slightly more hydrophobic than pseudomonic acid A, the most abundant product from the mupirocin synthesis pathway, the enhanced hydrophobicity consistent with a product lacking a hydroxyl.

