\chapter{General discussion}
\label{cha:chap7}

\section{Overview}
\label{sec:chap7Overview}
	The aim of the thesis was to explore details of the mupirocin biosynthesis pathway, focussing in particular on aspects of the function of the MmpA complex and its \textit{in trans} interactions. Generalizing some of the principles found should aid in the redesign of existing PKS, for the synthesis of novel compounds, with particular interest in medicinal compounds such as new antibiotics or potential anticancer agents. 
	
	In order to investigate the mechanism of MmpA this thesis focused mainly on exploiting already existing molecular modelling methods and proposes some protocols which can be used for similar purposes. Along with utilizing various pre existing molecular modelling methods the projects in this thesis heavily relied on Perl scripts written during the course of the project for various purposes for example HMM analysis and integrating GAFF parameters into GROMACS software. The computational protocols used along with the experimental validation of the predictions underlines the importance of molecular modelling in modern biological research, allowing the prediction of key residues in  MmpA and its interacting partner MupH that would have been expensive and very time consuming to discover solely by experiments. 
	
	At the inception of this thesis the major outstanding question was the \bet-branching mechanism in the mupirocin as well as other polyketide biosynthesis pathway. \bet-branching which is the addition of a methyl branch at the \bet \ position was hypothesised to be caused by the concerted action of five proteins collectively called the HCS cassette. Out of the five proteins an HMG-CoA homologue (MupH in the mupirocin system) protein is the first to interact with an ACP carrying the substrate to be \bet-branched. It was not understood what MupH recognizes on the C-terminal ACPs of MmpA. There are 16 ACPs associated with the PKSs and 5 discrete ACPs associated with the tailoring proteins in the mupirocin cluster. Out of the 21 ACPs how does MupH recognizes its partner ACP?  According to the position of the \bet-branch in the monic acid moiety it was easy to deduce that the \bet-branching happens at the end of the third module of MmpA. However, it does not tell us how does MupH exclude all the 21 possible partner APCs and only choose the two tandem ACPs in the MmpA.
	
	Sequence analysis carried out by Dr. Tony Haines discovered the presence of a conserved tryptophan six residues downstream of the catalytic cysteine in the ACPs associated with the \bet-branching but never conserved in the non-branching ACPs. Structure determination of the ACP-mupA3ab didomain by Dr. Matt Crump indicated that the tryptophan is buried at the core of the ACPs surrounded by the other highly conserve hydrophobic residues. Cross complementation experiments in Prof. Thomas' group showed that the branching ACPs from the thiomarinol system complements branching ACPs in the mupirocin system but that the non-branching ACPs do not. These experiments established that the branching ACPs differ from the non-branching ACPs and probably the structural variations between the two helps MupH in recognizing the correct ACP. However, a mechanistic view of what is happening at the interface of the two proteins when they interact was missing. Computational studies which included docking, molecular dynamics, hidden Markov model analysis, and sequence and structure based interface prediction carried out in this regard proposed the importance of the tryptophan at the core of the protein in maintaining the correct packaging of the core. This correct packaging would lead to the correct orientation of the helix III which forms the interface of the ACP with MupH. One residue mutation on the helix III, Y62F/A showed the importance of helix III at the interface. 
	
	Although the predicted ACP:MupH complex laid down the general rules for the \bet-branching ACP and HCS interaction there are more residues involved at the interface which define pair wise specificity between the ACP:HCS pairs in the different systems. Cross complementation experiments showed that BatC (the MupH homologue from kalimantacin system) failed to complement MupH in the mupirocin cluster. A sequence analysis revealed one residue near the ACP:MupH interface, that differs between the TmlH and MupH, which complements $ \Delta $\textit{mupH} \textit{in trans} and BatC that does not. The L to M mutation was found to lead to gain of function with BatC L219M complements $ \Delta $\textit{mupH}.
	
	Another way to resume the failed complementation of  $ \Delta $\textit{mupH} by BatC might be to swap the mupirocin ACPs for the cognate ACPs of BatC from the kalimantacin system. Suicide mutagenesis experiments were conducted to express the \bet-branching ACPs from the kalimantacin cluster in the mupirocin cluster. No pseudomonic acid production was detected in the HPLC traces for the strains expressing kalimantacin ACPs and the wild type MupH. This observation was in line with the initial hypothesis. However, upon expressing kalimantacin ACPs with the wild type BatC there was also no pseudomonic acid production detected. However, a new peak was detected which may correspond to a new metabolite being produced similar to the pseudomonic acids and experiments are on going to characterise it. 
	
	An alternative method for investigating pairwise specificity might be to make changes in the ACP-mupA3ab \bet-branching ACPs native to the mupirocin cluster, to allow them to function efficiently with the \textit{batC} when expressed \textit{in trans}, as previous experiments have shown the suitability of BatC in the mupirocin cluster. Molecular dynamics simulation of the ACP-mupA3a:MupH complex helped to refine the docking interface proposing residues at the interface which might play role in improving ACP-mupA3a's ability to interact well with the BatC. However, comparing the sequences of ACPs from the clusters which are likely to complement MupH and the ones which do not, there were no obvious patterns that identified putative key residues. 
	
	The docking experiments of ACP-mupA3a and MupH have shown that in order to reach the ligand in the correct orientation for the \bet-branching almost all of the phosphopantetheine along with the 16C monic acid precursor needs to get completely inside the MupH active site.  It has always been intriguing on how a large substrate like this is accommodated in the MupH active site. The docking analysis of rigid static structures does not tell us if there is an involvement of a dynamic element during the process. Molecular dynamics simulations on the ACP-mupA3a:MupH complex revealed a large movement in two loops at the surface of MupH at the opening of the MupH active site. Upon visualising the simulation trajectories it seems that these loop movement assist in the accommodation of the ligand in the active site. The distance between the mobile loops was found to be larger in the MupH monomer structures as compared to the MupH dimer structure. HMG-CoA orthologues exists as a dimer therefore its quite likely that MupH also exists as a dimer however, an experimental validation will be required to support this hypothesis. These large loop movements were not previously reported either through experiments or computational methods.
	
	Dynamic change in the structure upon ligand binding  was also identified in the ACPs. FAS ACPs are shown to sequester the acyl chains within their hydrophobic core through experiments as well as molecular dynamics simulations. However, no such observations had been made in the PKS ACPs. In order to explore this phenomenon in the PKS ACPs different apo, holo, and acyl forms of ACPs from the mupirocin cluster were simulated in explicit solvent for upto 1 $ \mu $s. The molecular dynamics simulations revealed that the PKS ACPs do form a cavity upon the attachment of the phosphopantetheine and acyl chains. The length of the acyl chain might also influence the size of the cavity. The cavity formed does not form a deep tunnel as in the FAS ACPs but, is rather a shallow, solvent exposed, surface groove, enabling the polar groups on the acyl chain to hydrogen bond with the solvent whilst shielding the hydrophobic parts of the polyketide.  It was also seen that a bulky residue at the proposed tunnel opening in the PKS ACPs prohibits the formation of a deep tunnel as opposed to the smaller residue at the equivalent position in the FAS ACPs. These dynamic behaviours were seen for the first time in the PKS ACPs in this work. %As explained earlier the ACP:MupH interaction can be exploited to either add or remove a \bet-branch similar interaction of ACPs with other proteins in the PKSs can be utilized to add diversity in the molecule. And understanding the dynamic behaviour of the PKS ACPs would help to re-engineer the interaction of ACPs with the myriad of domains involved in the polyketide synthesis.
	
	In re-engineering polyketides there are two major problems, the recognition specificity between the protein domains and the recognition specificity of the substrate. In the present work I worked on analysing the KS substrate specificity for an $ \alpha $-hydroxylated polyketide substrate by molecular docking of the cognate substrate of KS-mupA2 with the KS-mupA2 homo dimer. Dr. Joanne Hothersall knocked out MupA, which is thought to hydroxylate pseudomonic acid at the 6-OH position. For $ \Delta $\textit{mupA} the longest intermediate found was mupric acid which is produced by MmpD. This observation led to the hypothesis that an un $ \alpha $-hydroxylated substrate cannot be recognized by the KS-mupA2, the first condensing domain in the MmpA, resulting in the pathway shutdown. Docking the expected $ \alpha $-hydroxylated cognate substrate to the KS-mupA2 homo dimer revealed a loop in the opposite monomer at the dimer interface interacting with the $ \alpha $-OH. A sequence alignment of this loop from KS-mupA2 with all the other KSs from the mupirocin and thiomarinol cluster showed a similar loop in the equivalent KS from the thiomarinol system but no conservation was found with the other KSs. With this observation it was hypothesised that replacing this loop in KS-mupA2 with the equivalent loops from the KSs preceding and following it, KS-mupA1 and KS-mupA3, for which the cognate substrates does not have an $ \alpha $-OH, might allow the pathway to proceed further. By the time of writing this thesis Miss Yousra Alsamarraie from Prof. Thomas group was able to replace the KS-mupA2 loop with KS-mupA1 loop. The HPLC trace showed no peak for pseudomonic acid A or B but a new peak with larger retention time suggested that the pathway has produced a full length substrate, that is slightly more hydrophobic than pseudomonic acid A, consistent with dehydroxylated form of the product. More detailed structural analysis of the product is awaited, but the experiments do seem to show that these loops are important for specificity.

%\section{Future work}
%\label{sec:chap7FutureWork}	
%	\subsection{HMM analysis for sub-grouping ACPs on the basis of their cognate HCS type}	
%	\label{sec:acpHcsHmm}
%	Multiple sequence alignment and ACP:MupH modelling has not found indicated residues in the ACP that might define a subtype of residues associated with ACP:MupH interaction. A more sophisticated method like HMMs could be carried out to subgroup ACPs according to their cognate HCS protein type. This would enable the prediction of the mutations required to shift an ACP from one subgroup to another (using the method described in Section \ref{sec:MinumChanges}). Such an experiment would allow us to understand the pairwise specificity involved between ACP and HCS proteins in the \bet-branching mechanism. 
%	
%	\subsection{Molecular dynamics simulations of further ACP structures}
%	\label{sec:mdOnOtherAcps}
%	Molecular dynamic simulations carried on apo, holo and acyl forms of mupirocin ACPs revealed that the PKS ACPs can form a cavity upon the attachment of the phosphopantetheine and acyl chains similar to FAS ACPs. However, in the lieu of any experimental data available on the PKS ACPs another set of MD simulations in a different PKS system for example the ACP from module 2 of DEBS system (PDB ID 2JU1) and the ACP from curacin system (PDB ID 2LIU) could be performed to verify/extend the observations. It would also be interesting to test whether an FAS ACP would envelope a polyketide molecule into its  hydrophobic core in lieu of a fatty acid acyl chain. These studies might explain the dynamic mechanism through which small proteins like an ACPs recognize/interact with various domains in complex PKS machinery. The information gathered from such studies not only helps to gain a deeper understanding of protein structure and function in general but should also help to re-engineer PKS pathways for the production of novel compounds.	
%	
%	\subsection{Point mutations in PKS ACPs for investigating tunnel formation}
%	\label{sec:acpTunnelMutations}
%	Structural and sequence comparison of FAS and PKS ACPs revealed a bulky residue, isoleucine at the position 61 in the ACP-mupA3a  and a bulky residue at the position X in the GXDS motif, which seem to block the space equivalent to the ligand binding tunnel in the FAS ACP structure. It would be interesting to mutate position I61 and/or X in the GXDS motif to alanine in the ACP-mupA3a and possibly monitor a deeper transition of the acyl chain through NMR or further simulations. 
%	
%	\subsection{Identifying key residues determining KS specificity by point mutation}
%	\label{sec:ksLoopPointMutations}
%	The KS-mupA2 loop swap experiment have supported the dimer interface loop in KS-mupA2 as providing specificity for an $ \alpha $-hydroxylated substrate. It would now be interesting to carry out point mutations on the residues in the loop to pin point the key residue(s) responsible for the specificity of the $ \alpha $-hydroxylated substrates. It could also be beneficial to carry out an all atom molecular dynamics simulation on the whole complex in order to refine the KS dimer interface and map any favourable interaction between the docked ligand and the residue(s) in the specificity loop. %These point mutations would allow us not only to make a pathway pass through the KSs specific for an $ \alpha $-hydroxylated substrate but would also allow us to produce the un $ \alpha$-hydroxylated analogue in higher amounts. 
%	
%	\subsection{Carrying out sequence analysis to associate substrates based on the KS recognition loop}
%	\label{sec:substrateLoop}
%	 In the \textit{trans}-AT PKSs, \textcite{Nguyen2008} showed a strong correlation between the KS sequences and their preferred substrates. With the specificity for the substrates following the clades of a cladogram, \textcite{Jenner2013} showed the substrate specificity of a \textit{trans}-AT KS associated with the clade of \bet-branched substrates. A similar analysis could be carried out by comparing this recognition loop region from different systems and mapping the substrates associated with them. It could be possible that as the KS-mupA2 from the mupirocin system shared the same motif on the loop with the equivalent KS in the thiomarinol system, a clustering analysis would group KSs according to their recognition loops associated with a substrate type. This would in principle enable us to design mutagenesis experiments to re-engineer KSs by replacing the recognition loop with the type of substrate we want to be recognized.	 
%	 
%	\subsection{Building a model of module 5 of MmpA based on the KS-ACP docking complex and other available data}
%	\label{sec:mmpaRebuilding}
%	Till the year 2014 there was no structure available for a complete PKSs module. Two KS-AT di domains structures from the DEBS system and the mammalian FAS structure were used to serve as the scaffold for modelling PKSs from other systems. It was generally accepted that the module of a type I modular PKS would resemble the mammalian FAS structure owing to the similarity in the domain arrangement of the DEBS KS-AT didomain and the mammalian FAS structure. However, a cryo-EM structure published in 2014 by \textcite{Dutta2014} showed a very different scaffold for the complete module of type I PKS from pikromycin cluster, from the previously presumed domain architecture for the type I \textit{cis}-AT PKSs. There is also no complete module structure available for the \textit{trans}-AT PKSs. The only information available for the \textit{trans}-AT PKSs which could potentially be extended to model a complete module is the homodimeric KS structure with the post KS linker regions from the bacillaene cluster published by \textcite{Gay2014}. The variations observed between the cryo-EM structure from the pikromycin cluster and the previously presumed structure for type I \textit{cis}-AT PKSs suggests that the PKS modules in different systems may fold into different conformations and only share few overall similarities. These variations in conformations will lead to different interacting interfaces and different recognition specificity between the domains in the PKSs. Hence, in order to exploit effectively the recognition specificity between different domains in the PKSs it becomes necessary to deduce the complete modular structure either through experiments or through computational modelling. 
%	
%	Here, the docked complex of KS-mupA2 dimer with its cognate substrate and the ACP in the conformation which represents decarboxylation stage of the Claisen condensation reaction mechanism can be utilized to extend it further by docking KR and AT domains to the starting structure. The crystal structure of the \textit{trans}-AT KS dimer by \textcite{Gay2014} can be utilized to place the post KS linker domain correctly which could serve as a docking site for the \textit{trans} AT domain. Mapping rvET and PIER values should help to verify the correct placement of the interacting interface. Once a complete docked model is produced the whole complex can be subjected to a coarse grained simulation for example utilizing the MARTINI force field \parencite{Monticelli2008} in order to let the domains arrange themselves in an energetically favourable arrangement. PIER and rvET values can again be used to verify any improvements made at the interacting interfaces due to coarse grained simulations. 
%		
%%	\subsection{Carrying out molecular dynamics simulations on the wild type and C115Acetylated-cysteine MupH dimer structures}
%%	\label{sec:muphDimerSimulations}
%	
%	\subsection{Understanding protein-protein and protein-substrate specificity may lead to the creation of an automated platforms for synthetic biology purposes}
%	\label{sec:pipelineSyntheticBiology}
%	The advancements in the understanding of protein-protein and protein-substrate specificity involved in the PKS domains should lead to the development of automated platforms to predict the combination of domains required to produce a molecule of desired structure. At present there are many webservers that detect the PKS domains in a data (Table \ref{tab:pks_res}). Prediction of the metabolite produced by a PKS gene cluster is also possible based on domain function and specificity, for example the extender unit specificity of AT domains in a \textit{cis}-AT systems. However, there is no tool available that predict a likely expression system with the combination of PKS domains and tailoring enzymes to produce a molecule of desired structure. In terms of re-engineering PKSs most of the PKS researchers till now have concentrated on re-engineering their choice of system for producing a closely related molecule and most of these studies were done as a proof of concept rather than an attempt to produce a molecule with novel properties.  
%	 
%	In principle using a computer program it could be possible to break down the structure of a desired molecule into small moieties or allow a user to construct a molecule using the predefined building blocks which represent one condensation cycle or a post condensation modification. Using the knowledge we have about protein-protein and protein-substrate specificity various computational models can be trained to associate these building blocks with the domains/modules which can be combined together to produce the molecule of desired structure. Once an assembly line is predicted a synthetic DNA cluster could be created which can be expressed in a suitable host. 
	
\section{Conclusions}
\label{sec:conclusions}
The body of the work here shows the effectiveness of a range of modelling and bioinformatics techniques for the analysis of the protein-protein, domain-domain and protein-substrate interaction involved in polyketide synthesis. Combined with experiments performed by me and others elucidated key details of control of substrate flow at the start of the MmpA subunit from mupirocin system and of the \bet-methylation step at the end of the MmpA subunit. The methods used here and the results obtained should be applicable to other systems. In the future this may lead to experimental and computational tools to design and synthesis novel compounds de novo.

	
	
	